%% BioMed_Central_Tex_Template_v1.06
%%                                      %
%  bmc_article.tex            ver: 1.06 %
%                                       %

%%IMPORTANT: do not delete the first line of this template
%%It must be present to enable the BMC Submission system to
%%recognise this template!!

%%%%%%%%%%%%%%%%%%%%%%%%%%%%%%%%%%%%%%%%%
%%                                     %%
%%  LaTeX template for BioMed Central  %%
%%     journal article submissions     %%
%%                                     %%
%%          <8 June 2012>              %%
%%                                     %%
%%                                     %%
%%%%%%%%%%%%%%%%%%%%%%%%%%%%%%%%%%%%%%%%%


%%%%%%%%%%%%%%%%%%%%%%%%%%%%%%%%%%%%%%%%%%%%%%%%%%%%%%%%%%%%%%%%%%%%%
%%                                                                 %%
%% For instructions on how to fill out this Tex template           %%
%% document please refer to Readme.html and the instructions for   %%
%% authors page on the biomed central website                      %%
%% http://www.biomedcentral.com/info/authors/                      %%
%%                                                                 %%
%% Please do not use \input{...} to include other tex files.       %%
%% Submit your LaTeX manuscript as one .tex document.              %%
%%                                                                 %%
%% All additional figures and files should be attached             %%
%% separately and not embedded in the \TeX\ document itself.       %%
%%                                                                 %%
%% BioMed Central currently use the MikTex distribution of         %%
%% TeX for Windows) of TeX and LaTeX.  This is available from      %%
%% http://www.miktex.org                                           %%
%%                                                                 %%
%%%%%%%%%%%%%%%%%%%%%%%%%%%%%%%%%%%%%%%%%%%%%%%%%%%%%%%%%%%%%%%%%%%%%

%%% additional documentclass options:
%  [doublespacing]
%  [linenumbers]   - put the line numbers on margins

%%% loading packages, author definitions

%\documentclass[twocolumn]{bmcart}% uncomment this for twocolumn layout and comment line below
\documentclass{bmcart}
\usepackage{setspace} % Lindsay added for double-spacing
\usepackage{lineno} % Lindsay added for line numbering

%%% Load packages
%\usepackage{amsthm,amsmath}
%\RequirePackage{natbib}
%\RequirePackage[authoryear]{natbib}% uncomment this for author-year bibliography
%\RequirePackage{hyperref}
\usepackage[utf8]{inputenc} %unicode support
%\usepackage[applemac]{inputenc} %applemac support if unicode package fails
%\usepackage[latin1]{inputenc} %UNIX support if unicode package fails
\usepackage{makecell} % Lindsay added for line breaks in tables

%%%%%%%%%%%%%%%%%%%%%%%%%%%%%%%%%%%%%%%%%%%%%%%%%
%%                                             %%
%%  If you wish to display your graphics for   %%
%%  your own use using includegraphic or       %%
%%  includegraphics, then comment out the      %%
%%  following two lines of code.               %%
%%  NB: These line *must* be included when     %%
%%  submitting to BMC.                         %%
%%  All figure files must be submitted as      %%
%%  separate graphics through the BMC          %%
%%  submission process, not included in the    %%
%%  submitted article.                         %%
%%                                             %%
%%%%%%%%%%%%%%%%%%%%%%%%%%%%%%%%%%%%%%%%%%%%%%%%%


\def\includegraphic{}
\def\includegraphics{}



%%% Put your definitions there:
\startlocaldefs
\endlocaldefs


%%% Begin ...
\begin{document}

%%% Start of article front matter
\begin{frontmatter}

\begin{fmbox}
\dochead{Research}

\title{Transcriptomic responses to diet quality and viral infection in Apis mellifera}

%%%%%%%%%%%%%%%%%%%%%%%%%%%%%%%%%%%%%%%%%%%%%%
%%                                          %%
%% Enter the authors here                   %%
%%                                          %%
%% Specify information, if available,       %%
%% in the form:                             %%
%%   <key>={<id1>,<id2>}                    %%
%%   <key>=                                 %%
%% Comment or delete the keys which are     %%
%% not used. Repeat \author command as much %%
%% as required.                             %%
%%                                          %%
%%%%%%%%%%%%%%%%%%%%%%%%%%%%%%%%%%%%%%%%%%%%%%

\author[
   addressref={aff1},                   % id's of addresses, e.g. {aff1,aff2}
   email={lrutter@iastate.edu}   % email address
]{\inits{LR}\fnm{Lindsay} \snm{Rutter}}
\author[
   addressref={aff6},
   email={bbonning@ufl.edu}
]{\inits{BB}\fnm{Bryony C.} \snm{Bonning}}
\author[
   addressref={aff2},
   email={dicook@monash.edu}
]{\inits{DC}\fnm{Dianne} \snm{Cook}}
\author[
   addressref={aff3,aff4},
   email={amytoth@iastate.edu}
]{\inits{AT}\fnm{Amy L.} \snm{Toth}}
\author[
   addressref={aff5},
   corref={aff5},
   email={adolezal@illinois.edu}
]{\inits{AD}\fnm{Adam} \snm{Dolezal}}
%%%%%%%%%%%%%%%%%%%%%%%%%%%%%%%%%%%%%%%%%%%%%%
%%                                          %%
%% Enter the authors' addresses here        %%
%%                                          %%
%% Repeat \address commands as much as      %%
%% required.                                %%
%%                                          %%
%%%%%%%%%%%%%%%%%%%%%%%%%%%%%%%%%%%%%%%%%%%%%%

\address[id=aff1]{%                           % unique id
  \orgname{Bioinformatics and Computational Biology Program, Iowa State University},
  \city{Ames},   
  \state{IA}
  \postcode{50011},
  \cny{USA}
}
\address[id=aff2]{%
  \orgname{Econometrics and Business Statistics, Monash University},
  \city{Clayton},   
  \state{VIC}
  \postcode{3800},
  \cny{Australia}
}

\address[id=aff3]{%
  \orgname{Department of Entomology, Iowa State University},
  \city{Ames},   
  \state{IA}
  \postcode{50011},
  \cny{USA}
}

\address[id=aff4]{%
  \orgname{Department of Ecology, Evolution, and Organismal Biology, Iowa State University},
  \city{Ames},   
  \state{IA}
  \postcode{50011},
  \cny{USA}
}

\address[id=aff5]{%
  \orgname{Department of Entomology, University of Illinois at Urbana-Champaign},
  \city{Urbana},   
  \state{IL}
  \postcode{61801},
  \cny{USA}
}

\address[id=aff6]{%
  \orgname{Department of Entomology and Nematology, University of Florida},
  \city{Gainesville},   
  \state{FL}
  \postcode{32611},
  \cny{USA}
}

%%%%%%%%%%%%%%%%%%%%%%%%%%%%%%%%%%%%%%%%%%%%%%
%%                                          %%
%% Enter short notes here                   %%
%%                                          %%
%% Short notes will be after addresses      %%
%% on first page.                           %%
%%                                          %%
%%%%%%%%%%%%%%%%%%%%%%%%%%%%%%%%%%%%%%%%%%%%%%

\begin{artnotes}
%\note{Sample of title note}     % note to the article
%\note[id=n1]{Equal contributor} % note, connected to author
\end{artnotes}

\end{fmbox}% comment this for two column layout

%%%%%%%%%%%%%%%%%%%%%%%%%%%%%%%%%%%%%%%%%%%%%%
%%                                          %%
%% The Abstract begins here                 %%
%%                                          %%
%% Please refer to the Instructions for     %%
%% authors on http://www.biomedcentral.com  %%
%% and include the section headings         %%
%% accordingly for your article type.       %%
%%                                          %%
%%%%%%%%%%%%%%%%%%%%%%%%%%%%%%%%%%%%%%%%%%%%%%

\begin{abstractbox}

\begin{abstract}
\parttitle{Background}
Parts of Europe and the United States have witnessed dramatic losses in commercially managed honey bees over the past decade to what is considered an unsustainable extent. The large-scale loss of honey bees has considerable implications for the agricultural economy because honey bees are one of the leading pollinators of numerous crops. Honey bee declines have been associated with several interactive factors. Poor nutrition and viral infection are two environmental stressors that pose heightened dangers to honey bee health. In this study, we used RNA-sequencing to examine how monofloral diets and Israeli acute paralysis virus inoculation influence gene expression patterns in honey bees.

\parttitle{Results}
We found a considerable nutritional response, with almost 2,000 transcripts changing with diet quality. The majority of these genes were over-represented for nutrient signaling (insulin resistance) and immune response (Notch signaling and JaK-STAT pathways). Somewhat unexpectedly, the transcriptomic response to viral infection was fairly limited. We only found 43 transcripts to be differentially expressed, some with known immune functions (argonaute-2), transcriptional regulation, and muscle contraction. We created contrasts to determine if any protective mechanisms of good diet were due to direct effects on immune function (resistance) or indirect effects on energy availability (tolerance). A similar number of resistance and tolerance candidate differentially expressed genes were found, suggesting both processes may play significant roles in dietary buffering from pathogen infection. We also compared the virus main effect in our study (polyandrous colonies) to that obtained in a previous study (single-drone colonies) and verified significant overlap in differential expression despite visualization methods showing differences in the noisiness levels between these two datasets.

\parttitle{Conclusions}
Through transcriptional contrasts and functional enrichment analysis, we add to evidence of feedbacks between diet and disease in honey bees. We also show that comparing results derived from polyandrous colonies (which are typically more natural) and single-drone colonies (which usually yield more signal) may allow researchers to identify transcriptomic patterns in honey bees that are concurrently less artificial and less noisy. Altogether, we hope this work underlines possible merits of using data visualization techniques and multiple datasets when interpreting RNA-sequencing studies.

\end{abstract}

%%%%%%%%%%%%%%%%%%%%%%%%%%%%%%%%%%%%%%%%%%%%%%
%%                                          %%
%% The keywords begin here                  %%
%%                                          %%
%% Put each keyword in separate \kwd{}.     %%
%%                                          %%
%%%%%%%%%%%%%%%%%%%%%%%%%%%%%%%%%%%%%%%%%%%%%%

\begin{keyword}
\kwd{Honey bee}
\kwd{RNA-sequencing}
\kwd{Israeli acute paralysis virus}
\kwd{Monofloral pollen}
\kwd{Visualization}
\end{keyword}

% MSC classifications codes, if any
%\begin{keyword}[class=AMS]
%\kwd[Primary ]{}
%\kwd{}
%\kwd[; secondary ]{}
%\end{keyword}

\end{abstractbox}
%
%\end{fmbox}% uncomment this for twcolumn layout

\end{frontmatter}

%%%%%%%%%%%%%%%%%%%%%%%%%%%%%%%%%%%%%%%%%%%%%%
%%                                          %%
%% The Main Body begins here                %%
%%                                          %%
%% Please refer to the instructions for     %%
%% authors on:                              %%
%% http://www.biomedcentral.com/info/authors%%
%% and include the section headings         %%
%% accordingly for your article type.       %%
%%                                          %%
%% See the Results and Discussion section   %%
%% for details on how to create sub-sections%%
%%                                          %%
%% use \cite{...} to cite references        %%
%%  \cite{koon} and                         %%
%%  \cite{oreg,khar,zvai,xjon,schn,pond}    %%
%%  \nocite{smith,marg,hunn,advi,koha,mouse}%%
%%                                          %%
%%%%%%%%%%%%%%%%%%%%%%%%%%%%%%%%%%%%%%%%%%%%%%

%%%%%%%%%%%%%%%%%%%%%%%%% start of article main body
% <put your article body there>

%%%%%%%%%%%%%%%%
%% Background %%
%%

\begin{linenumbers} % Lindsay added
\begin{doublespacing} % Lindsay added

\section*{Background}

Commercially managed honey bees have undergone unusually large declines in the United States and parts of Europe over the past decade \cite{ccd1, ccd2, ccd3}, with annual mortality rates exceeding what beekeepers consider sustainable \cite{ccd5, ccd6}. More than 70 percent of major global food crops (including fruits, vegetables, and nuts) at least benefit from pollination, and yearly insect pollination services are valued worldwide at \$175 billion \cite{ccd7}. As honey bees are largely considered to be the leading pollinator of numerous crops, their marked loss has considerable implications for agricultural sustainability \cite{ccd4}.

Honey bee declines have been associated with several factors, including pesticide use, parasites, pathogens, habitat loss, and poor nutrition \cite{factors, factors2}. Researchers generally agree that these stressors do not act in isolation; instead, they appear to influence the large-scale loss of honey bees in an interactive fashion as the environment changes \cite{interacting}. Nutrition and viral infection are two broad factors that pose heightened dangers to honey bee health in response to recent environmental changes.

Pollen is a main source of nutrition (including proteins, amino acids, lipids, sterols, starch, vitamins, and minerals) in honey bees \cite{source, source2}. At the individual level, pollen supplies most of the nutrients necessary for physiological development \cite{brodschneider} and is believed to have considerable impact on longevity \cite{longevity}. At the colony level, pollen enables young workers to produce jelly, which then nourishes larvae, drones, older workers, and the queen \cite{jelly, jelly2}. Various environmental changes (including urbanization and monoculture crop production) have significantly altered the nutritional profile available to honey bees. In particular, honey bees are confronted with a less diverse selection of pollen, which is of concern because mixed-pollen (polyfloral) diets are generally considered healthier than single-pollen (monofloral) diets \cite{diverse, diverse2, alaux}. Indeed, reported colony mortality rates are higher in developed land areas compared to undeveloped land areas \cite{undeveloped}, and beekeepers rank poor nutrition as one of the main reasons for colony losses \cite{bkLoss}. Understanding how undiversified diets affect honey bee health will be crucial to resolve problems that may arise as agriculture continues to intensify throughout the world \cite{ag, ag2}.

Viral infection was a comparatively minor problem in honey bees until the last century when \textit{Varroa destructor} (an ectoparasitic mite) spread worldwide \cite{miteSpread}. This mite feeds on honey bee hemolymph \cite{hemolymph}, transmits multiple viruses, and supports replication of some viruses \cite{miteVirus, miteVirus2, miteVirus3, newVR}. More than 20 honey bee viruses have been identified \cite{numVirus}. One of these viruses that has been linked to honey bee decline is Israeli acute paralysis virus (IAPV). A positive-sense RNA virus of the family Dicistroviridae \cite{fam}, IAPV infection causes shivering wings, decreased locomotion, muscle spams, paralysis, and high premature death percentages in caged infected adult honey bees \cite{symptoms}. IAPV has demonstrated higher infectious capacities than other honey bee viruses under certain conditions \cite{carrillo} and is more prevalent in colonies that do not survive the winter \cite{winter}. Its role in the rising phenomenon of ``Colony Collapse Disorder'' (in which the majority of worker bees disappear from a hive) remains unclear: It has been implicated in some studies \cite{iapvCCD, iapvCCD2} but not in other studies \cite{ccd1, fam, iapvCCD3}. Nonetheless, it is clear that IAPV reduces colony strength and survival.

Although there is growing interest in how viruses and diet quality affect the health and sustainability of honey bees, as well as a recognition that such factors might operate interactively, there are only a small number of experimental studies thus far directed toward elucidating the interactive effects of these two factors in honey bees \cite{intNV, intNV2, intNV3}. We recently used laboratory cages and nucleus hive experiments to investigate the health effects of these two factors, and our results show the importance of the combined effects of both diet quality and virus infection. Specifically, ingestion by honey bees of high quality pollen is able to mitigate virus-induced mortality to the level of diverse, polyfloral pollen \cite{adamInt}. 

Following up on these phenotypic findings from our previous study, we now aim to understand the corresponding underlying mechanisms by which high quality diets protect bees from virus-induced mortality. For example, it is not known whether the protective effect of good diet is due to direct, specific effects on immune function (resistance), or if it is due to indirect effects of good nutrition on vigor (tolerance) \cite{resTol1}. Transcriptomics is one means to better understand the mechanistic underpinnings of dietary and viral effects on honey bee health. Transcriptomic analysis can help us identify 1) the genomic scale of transcriptomic response to diet and virus infection, 2) whether these factors interact in an additive or synergistic way on transcriptome function, and 3) the types of pathways affected by diet quality and viral infection. This information, heretofore lacking in the literature, can help us better understand how good nutrition may be able to serve as a ``buffer'' against other stressors \cite{AdamTothReview}.

As it stands, there are only a small number of published experiments examining gene expression patterns related to diet effects \cite{alaux2} and virus infection effects \cite{galbraith} in honey bees. Honey bee transcriptomic studies have found that pollen nutrition upregulates genes involved in macromolecule metabolism, longetivity, and the insulin/TOR pathway required for physiological development \cite{alaux2}. Insect gene expression studies have implicated RNA silencing, autophagy, JAK/STAT, Toll, and IMD as antiviral pathways for a range of viral infections, including dicistrovirus infections \cite{galbraith, Avadhanula, fewDEGs2, kemp, costa}. Transcriptional pausing is also believed to be imperative for early antiviral immunity in many insects \cite{resTol11}. See \cite{cherryReview} for a review of known antiviral mechanisms in insect models.

As far as we know, there are few to no studies investigating honey bee gene expression patterns specifically related to monofloral diets, and few to no studies investigating honey bee gene expression patterns related to the combined effects of diet in any broad sense and viral inoculation in any broad sense. In this study, we examine how monofloral diets and viral inoculation influence gene expression patterns in honey bees by focusing on four treatment groups (low quality diet without IAPV exposure, high quality diet without IAPV exposure, low quality diet with IAPV exposure, and high quality diet with IAPV exposure). We conduct RNA-sequencing analysis on a randomly selected subset of the honey bees we used in our previous study (as is further described in our methods section). We then examine pairwise combinations of treatment groups, the main effect of monofloral diet, the main effect of IAPV exposure, and the combined effect of the two factors on gene expression patterns.

We also compare the main effect of IAPV exposure in our dataset to that obtained in a previous study conducted by Galbraith and colleagues \cite{galbraith}. As RNA-sequencing data can be biased \cite{biased1, biased2, biased3}, this comparison allowed us to characterize how repeatable and robust our RNA-sequencing results were in comparison to previous studies. Importantly, we use an in-depth data visualization approach to explore and corroborate our data, and suggest such an approach can be useful for cross-study comparisons and validation of noisy RNA-sequencing data in the future.

\section*{Results}

\subsection*{Phenotypic results}

We reanalyzed our previously published dataset with a subset that focuses on diet quality and is more relevant to the current study. We briefly show it again here to inform the RNA-sequencing comparison because we reduced the number of treatments from the original published data (from eight to four) \cite{adamInt} as a means to focus on diet quality effects.

As shown in Figure 1, mortality rates of honey bees 72 hour post-inoculation significantly differed among the treatment groups (mixed model ANOVA across all treatment groups, df = 3, 54; F = 10.03; p $<$ 2.34e-05). The effect of virus treatment (mixed model ANOVA, df = 1, 54; F = 24.73; p $<$ 7.04e-06) and diet treatment (mixed model ANOVA, df = 1, 54; F = 5.32; p $<$ 2.49e-02) were significant, but the interaction between the two factors (mixed model ANOVA, df = 1, 54; F = 4.72e-02, p = 8.29e-01) was not significant. We compared mortality levels based on pairwise comparisons: For a given diet, honey bees exposed to the virus showed significantly higher mortality rate than honey bees not exposed to the virus. Namely, bees fed Rockrose pollen had significantly elevated mortality with virus infection compared to uninfected controls (Benjamini-Hochberg, p $<$ 1.53e-03), and bees fed Chestnut pollen similarly had significantly elevated mortality with virus infection compared to controls (Benjamini-Hochberg, p $<$ 3.12e-03) (Figure 1).

As shown in Figure 2, IAPV titers of honey bees 72 hour post-inoculation significantly differed among the treatment groups (mixed model ANOVA across all treatment groups, df = 3, 33; F = 6.10; p $<$ 2.03e-03). The effect of virus treatment (mixed model ANOVA, df = 1, 33; F = 15.04; p $<$ 4.75e-04) was significant, but the diet treatment (mixed model ANOVA, df = 1, 33; F = 2.55; p = 1.20e-01) and the interaction between the two factors (mixed model ANOVA, df = 1, 33; F = 7.02e-01, p = 4.08e-01) were not significant. We compared IAPV titers  based on pairwise comparisons: Bees fed Rockrose pollen had significantly elevated IAPV titers with virus infection compared to uninfected controls (Benjamini Hochberg, p $<$ 7.56e-03). However, bees fed Chestnut pollen did not have significantly elevated IAPV titers with virus infection compared to uninfected controls (Benjamini Hochberg, p = 6.29e-02). Overall, we interpreted these findings to mean that high-quality Chestnut pollen could ``rescue'' high virus titers resulting from the inoculation treatment, whereas low-quality Rockrose pollen could not (Figure 2).

\subsection*{Main effect DEG results}

We observed a substantially larger number of differentially expressed genes (DEGs) in our diet main effect (\textit{n} = 1,914) than in our virus main effect (\textit{n} = 43) (Supplementary table 1 A and B, Additional file 1). In the diet factor, more DEGs were upregulated in the more-nutritious Chestnut group (\textit{n} = 1,033) than in the less-nutritious Rockrose group (\textit{n} = 881). In the virus factor, there were more virus-upregulated DEGs (\textit{n} = 38) than control-upregulated DEGs (\textit{n} = 5). While these reported DEG counts are from the DESeq2 package, we saw similar trends for the edgeR and limma package results (Supplementary table 1, Additional file 1 and Additional file 18).

GO analysis of the Chestnut-upregulated DEGs revealed the following over-represented categories: Wnt signaling, hippo signaling, and dorso-ventral axis formation, as well as pathways related to circadian rhythm, mRNA surveillance, insulin resistance, inositol phosphate metabolism, FoxO signaling, ECM-receptor interaction, phototransduction, Notch signaling, JaK-STAT signaling, MAPK signaling, and carbon metabolism (Supplementary table 2, Additional file 1). GO analysis of the Rockrose DEGs revealed pathways related to terpenoid backbone biosynthesis, homologous recombination, SNARE interactions in vesicular transport, aminoacyl-tRNA biosynthesis, Fanconi anemia, and pyrimidine metabolism (Supplementary table 3, Additional file 1).

With so few DEGs (\textit{n} = 43) in our virus main effect comparison, we focused on individual genes and their known functionalities rather than GO over-representation (Table \ref{tbl:virusGenes}). Of the 43 virus-related DEGs, only 10 had GO assignments within the DAVID database. These genes had putative roles in the recognition of pathogen-related lipid products and the cleaving of transcripts from viruses, as well as involvement in ubiquitin and proteosome pathways, transcription pathways, apoptotic pathways, oxidoreductase processes, and several more functions (Table \ref{tbl:virusGenes}).

No interaction DEGs were observed between the diet and virus factors of the study, in any of the pipelines (DESeq2, edgeR, and limma).

\subsection*{Pairwise comparison of DEG results}

The number of DEGs across the six treatment pairings between the diet and virus factor ranged from 0 to 955 (Supplementary table 8, Additional file 1). Some of the trends observed in the main effect comparisons persisted: The diet level appeared to have greater influence on the number of DEGs than the virus level. Across every pair comparing the Chestnut and Rockrose levels, regardless of the virus level, the number of Chestnut-upregulated DEGs was higher than the number of Rockrose-upregulated DEGs (Supplementary table 8 C, D, E, F, Additional file 1). For the pairs in which the diet level was controlled, the virus-exposed treatment showed equal to or more DEGs than the control treatment (Supplementary table 8 A and B, Additional file 1). There were no DEGs between the treatment pair controlling for the Chestnut level of the virus effect (Supplementary table 8A, Additional file 1). These trends were observed for all three pipelines used (DESeq2, edgeR, and limma).

\subsection*{Prior study comparison results}

We wished to explore the signal:to:noise ratio between the Galbraith dataset and our dataset. Note that the Galbraith dataset contained three samples for each virus level, while our dataset contained twelve samples for each virus level. Basic PCA plots were constructed with the DESeq2 analysis pipeline and showed that the Galbraith dataset may separate the infected and uninfected honey bees better than our dataset (Additional file 2). We also noted that the first replicate of both treatment groups in the Galbraith data did not cluster as cleanly in the PCA plots. However, through this automatically-generated plot, we can only visualize information at the sample level. Wanting to learn more about the data at the gene level, we continued with additional visualization techniques.

We used parallel coordinate lines superimposed onto boxplots to visualize the DEGs associated with virus infection in the two studies. The background side-by-side boxplot represents the distribution of all genes in the data, and each parallel coordinate line represents one DEG. To reduce overplotting of parallel coordinate lines, we used hierarchical clustering techniques to separate DEGs into common patterns as is described in the methods section.

We see that the 1,019 DEGs from the Galbraith dataset form relatively clean-looking visual displays (Figure 3). We do see that the first replicate of the virus group (V.1) appears somewhat inconsistent with the other virus replicates in Cluster 1, confirming that this trend in the data that we saw in the PCA plot carried through into the DEG results. In contrast, we see that the 43 virus-related DEGs from our dataset do not look as clean in their visual displays (Figure 4). The replicates appear somewhat inconsistent in their estimated expression levels and there is not always such a large difference between treatment groups. We see a similar finding when we also examine a larger subset of 1,914 diet-related DEGs from our study (Additional file 3).

We also used litre plots to examine the structure of individual DEGs: We see that indeed the individual virus DEGs from our data (Additional file 4) show less consistent replications and less differences between the treatment groups compared to the individual virus DEGs from the Galbraith data (Additional files 5 and 6). For the Galbraith data, we examined individual DEGs from the first cluster (Additional file 5) and second cluster (Additional file 6) because the first cluster had previously shown less consistency in the first replicate of the treatment group (Figure 3). We verify this trend again in the litre plots with the DEG points in the first cluster showing less tight cluster patterns (Additional files 5 and 6).

Finally, we looked at scatterplot matrices to assess the DEGs. We created standardized scatterplot matrices for each of the four clusters (from Figure 3) of the Galbraith data (Additional files 7, 8, 9, and 10). We also created standardized scatterplot matrices for our data. However, as our dataset contained 24 samples, we would need to include 276 scatterplots in our matrix, which would be too numerous to allow for efficient visual assessment of the data. As a result, we created four scatterplot matrices of our data, each with subsets of 6 samples to be more comparable to the Galbraith data (Additional files 11, 12, 13, and 14). We can again confirm through these plots that the DEGs from the Galbraith data appeared more as expected: They deviated more from the \textit{x=y} line in the treatment scatterplots while staying close to the \textit{x=y} line in replicate scatterplots.

Despite the virus-related DEGs (\textit{n} = 1,019) from the Galbraith dataset displaying the expected patterns more than those from our dataset (\textit{n} = 43), there was significant overlap (p-value $<$ 2.2e-16) in the DEGs between the two studies, with 26/38 (68\%) of virus-upregulated DEGs from our study also showing virus-upregulated response in the Galbraith study (Figure 6).

\subsection*{Tolerance versus resistance results}

Using the contrasts specified in Table \ref{tbl:contrasts}, we discovered 122 ``tolerance'' candidate DEGs and 125 ``resistance'' candidate DEGs. We again used parallel coordinate lines superimposed onto side-by-side boxplots to visualize these candidate DEGs. To reduce overplotting of parallel coordinate lines, we again used hierarchical clustering techniques to separate DEGs into common patterns. Perhaps unsurprisingly, we still see a substantial amount of noise (inconsistency between replicates) in our resulting candidate DEGs (Additional files 15 and 16). However, the broad patterns we expect to see still emerge: For example, based on the contrasts we created to obtain the `tolerance'' candidate DEGs, we expect them to display larger count values in the ``NC'' group compared to the ``NR'' group and larger count values in the ``VC'' group compared to the ``VR'' group. Indeed, we see this pattern in the associated parallel coordinate plots (Additional file 15). Likewise, based on the contrasts we created to obtain the `resistance'' candidate DEGs, we still expect them to display larger count values in the ``VC'' group compared to the ``VR'' group, but we no longer expect to see a difference between the ``NC'' and ``NR'' groups. We do generally see these expected patterns in the associated parallel coordinate plots: While there are large outliers in the ``NC'' group, the ``NR'' replicates are no longer typically below a standardized count of zero (Additional file 16). The genes in Cluster 3 may follow the expected pattern the most distinctively (Additional file 16).

Within our 122 ``tolerance'' gene ontologies, we found functions related to metabolism (such as carbohydrate metabolism, fructose metabolism, and chitin metabolism). However, we also discovered gene ontologies related to RNA polymerase II transcription, immune response, and regulation of response to reactive oxygen species (Figure 5A). Within our 125 ``resistance'' gene ontologies, we found functions related to metabolism (such as carbohydrate metabolism, chitin metabolism, oligosaccharide biosynthesis, and general metabolism) (Figure 5B).

\subsection*{Post hoc analysis results}

In general, the R-squared values between gene read counts and pathogen response measurements were low (R-squared $<$ 0.1). However, some DEG clusters showed slightly larger R-squared values than the non-DEG group (the rest of the data). One prominent example of this includes the first and second cluster of the virus-related DEGs and their correlation with IAPV titers (Additional file 19I). The Kruskal–Wallis test was used to determine if R-squared populations of DEG clusters significantly differed from those in the rest of the data. The p-values and Bonferroni correction values for each of the 36 tests (as described in the methods section) is provided in Supplementary table 9, Additional file 1. An overall trend emerges to suggest that DEGs may have significantly larger correlation with the pathogen response measurements compared to non-DEGs. It is difficult to interpret these results in light of the noisiness of this data, but it may be of interest to conduct further studies examining differential expression between pathogen response measurements. 

\section*{Discussion}

Challenges to honey bee health are a growing concern, in particular the combined, interactive effects of nutritional stress and pathogens (Dolezal and Toth 2018). In this study, we used RNA-sequencing to probe mechanisms underlying honey bee responses to two effects, diet quality and infection with the prominent virus of concern, IAPV. In general, we found a major nutritional transcriptomic response, with nearly 2,000 transcripts changing in response to diet quality (rockrose/poor diet versus chestnut/good diet). The majority of these genes were upregulated in response to high quality diet, and these genes were over-represented for functions (Supplementary table 2, Additional file 1) such as nutrient signaling metabolism (insulin resistance) and immune response (Notch signaling and JaK-STAT pathways). These data suggest high quality nutrition may allow bees to alter their metabolism, favoring investment of energy into innate immune responses.

While some insect systems have shown relatively low transcriptional responses to dicistrovirus infection \cite{fewDEGs, fewDEGs2}, previous work on honey bees has revealed many hundreds of DEGs \cite{galbraith}. Despite this, the transcriptomic response to virus infection in our experiment was fairly limited. We found only 43 transcripts to be differentially expressed, some with known immune functions (Table \ref{tbl:virusGenes}) such as argonaute-2 and a gene with similarity to MD-2 lipid recognition protein, as well as genes related to transcriptional regulation and muscle contraction. The small number of DEGs in this study may be partly explained by the large amount of noise in the data (Figure 4 and Additional files 2B, 4, 11, 12, 13, and 14).

Given the noisy nature of our data, and our desire to hone in on genes with real expression differences, we compared our data to the Galbraith study \cite{galbraith}, which also examined bees response to IAPV infection. In contrast to our study, Galbraith et al. identified a large number of virus responsive transcripts, and generally had less noise in their data (Figure 3 and Additional files 2A, 5, 6, 7, 8, 9, and 10). To identify the most consistent virus-responsive genes from our study, we looked for overlap in the DEGs associated with virus infection on both experiments. We found a large, statistically significant (p-value $<$ 2.2e-16) overlap, with 26/38 (68\%) of virus-responsive DEGs from our study also showing response to virus infection in Galbraith et al. (Figure 6). This result gives us confidence that, although noisy, we were able to uncover reliable, replicable gene expression responses to virus infection with our data.

Data visualization is a useful method to identify noise and robustness in RNA-sequencing data \cite{edger}. In this study, we used extensive data visualization to improve the interpretation of our RNA-sequencing results. For example, the DESeq2 package comes with certain visualization options that are popular in RNA-sequencing analysis. One of these visualization is the principal component analysis (PCA) plot, which allows users to visualize the similarity between samples within a dataset. We could determine from this plot that indeed the Galbraith data may show more similarity between its replicates and differences between its treatments compared to our data (Additional file 2). However, the PCA plot only shows us information at the sample level. We wanted to investigate how these differences in the signal:to:noise ratios of the datasets would affect the structure of any resulting DEGs. As a result, we also used three plotting techniques from the bigPint package: We investigated the 1,019 virus-related DEGs from the Galbraith dataset and the 43 virus-related DEGs from our dataset using parallel coordinate lines, scatterplot matrices, and litre plots. To prevent overplotting issues in our graphics, we used a hierarchical clustering technique for the parallel coordinate lines to separate the set of DEGs into smaller groups. We also needed to examine four subsets of samples from our dataset to make effective use of the scatterplot matrices. After these tailorizations, we determined that the same patterns we saw in the PCA plots regarding the entire dataset extended down the pipeline analysis into the DEG calls: Even the DEGs from the Galbraith dataset showed more similarity between their replicates and differences between their treatments compared to those from our data. However, the 365 DEGs from the Galbraith data in Cluster 1 of Figure 3 showed an inconsistent first replicate in the treatment group (``V.1''), which was something we observed in the PCA plot. This indicates that this feature also extended down the analysis pipeline into DEG calls. Despite the differences in signal between these two datasets, there was substantial overlap in the resulting DEGs. We believe these visualization applications can be useful for future researchers analyzing RNA-sequencing data to quickly and effectively ensure that the DEG calls look reliable or at least overlap with DEG calls from similar studies that look reliable. We also expect this type of visualization exploration can be especially crucial when studying complex organisms that do not have genetic identicalness or similarity between replicates and/or when using experiments that may lack rigid design control.

One of the goals of this study was to use our RNA-sequencing data to assess whether transcriptomic responses to diet quality and virus infection provide insight into whether high quality diet can buffer bees from pathogen stress via mechanisms of ``resistance'' or ``tolerance''. Recent evidence has suggested that overall immunity is determined by more than just ``resistance'' (the reduction of pathogen fitness within the host by mechanisms of avoidance and control) \cite{resTol2}. Instead, overall immunity is related to ``resistance'' in conjunction with ``tolerance'' (the reduction of adverse effects and disease resulting from pathogens by mechanisms of healing) \cite{resTol1, resTol2}. Immune-mediated resistance and diet-driven tolerance mechanisms are costly and may compete with each other \cite{resTol1, resTol4}. Data and models have suggested that selection can favor an optimum combination of both resistance and tolerance \cite{resTol5, resTol6, resTol7, resTol8}. We attempted to address this topic through specific gene expression contrasts (Table \ref{tbl:contrasts}), accompanied by GO analysis of the associated gene lists. We found an approximately equal number of resistance (\textit{n} = 125) and tolerance (\textit{n} = 122) related candidate DEGs, suggesting both processes may be playing significant roles in dietary buffering from pathogen induced mortality. Resistance candidate DEGs had functions related to several forms of metabolism (chitin and carbohydrate), regulation of transcription, and cell adhesion (Figure 5B). Tolerance candidate DEGs had functions related to carbohydrate metabolism and chitin metabolism; however, they also showed functions related to immune response, including RNA polymerase II transcription (Figure 5A). Previous studies have shown that transcriptional pausing of RNA polymerase II may be an innate immune response in \textit{D. melanogaster} that allows for a more rapid response by increasing the accessibility of promoter regions of virally induced genes \cite{resTol11}. These possible immunological defense mechanisms within our ``tolerance'' candidate DEGs and metabolic processes within our ``resistance'' candidate DEGs may provide additional evidence of feedbacks between diet and disease in honey bees \cite{AdamTothReview}.

There were several limitations in this study that could be improved upon in future studies. For instance, our comparison between the Galbraith data (single-drone colonies) and our data (polyandrous colonies) was limited by numerous extraneous variables between these studies. In addition to different molecular pipelines and bioinformatic preprocessing pipelines used between these studies, the Galbraith study focused on one-day old worker honey bees that were fed sugar and artificial pollen diet, whereas our study focused on adult worker honey bees that were fed bee-collected monofloral diets. Furthermore, the Galbraith data used eviscerated abdomens with attached fat bodies and only considered symptomatic honey bees for their infected treatment group, whereas we used whole bodies and considered both asymptomatic and symptomatic honey bees for our infected treatment group. Further differences between the studies can be found in their corresponding published methods sections \cite{adamInt, galbraith}. Our comparative visualization assessment between these two datasets was also somewhat limited because the virus effect in the Galbraith study used three replicates for each level, whereas the virus effect in our study used twelve replicates for each level that were actually further subdivided into six replicates for each diet level. Hence the apparent reduction in noise observed in the Galbraith data compared to our data in the PCA plots, parallel coordinate plots, scatterplot matrices, and litre plots may be an inadvertent product of the smaller number of replicates used and the lack of a secondary treatment group rather than solely the reduction in genetic variability through the single-drone colony design itself. With this in mind, while our current efforts may be a starting point, future studies can shed more light on signal:to:noise and differential expression differences between polyandrous colony designs and single-drone colony designs by controlling for extraneous factors more strictly than what we were able to do in the current line of work. 

In addition, this study used a whole body RNA-sequencing approach. In future related studies, it may be informative to use tissue-specific methods. Previous work has shown that even though IAPV replication occurs in all honey bee tissues, it localizes more in gut and nerve tissues and in the hypopharyngeal glands. Likewise, the highest IAPV titers have been observed in gut tissues \cite{winter}. Recent evidence has suggested that RNA-sequencing approaches toward composite structures in honey bees leads to false negatives, implying that genes strongly differentially expressed in particular structures may not reach significance within the composite structure \cite{tissueLevel}. These studies have also found that within a composite extraction, structures therein may contain opposite patterns of differential expression. We can provide more detailed answers to our original transcriptomic questions if we were to repeat this same experimental design only now at a more refined tissue level. Another future direction related to this work would be to integrate multiple omics datasets to investigate monofloral diet quality and IAPV infection in honey bees. Indeed, previous studies in honey bees have found that multiple omics datasets do not always align in a clear-cut manner, and hence may broaden our understanding of the molecular mechanisms being explored \cite{galbraith}.

\section*{Conclusions}

To the best of our knowledge, there are few to no studies investigating honey bee gene expression specifically related to monofloral diets, and few to no studies examining honey bee gene expression related to the combined effects of diet in any general sense and viral inoculation in any general sense. It also remains unknown whether the protective effects of good diet in honey bees is due to direct effects on immune function (resistance) or indirect effects of energy availability on vigor and health (tolerance). We attempted to address these unresolved areas by conducting a two-factor RNA-sequencing study that examined how monofloral diets and IAPV inoculation influence gene expression patterns in honey bees. Overall, our data suggest complex transcriptomic responses to multiple stressors in honey bees. Diet has the capacity for large and profound effects on gene expression and may set up the potential for both resistance and tolerance to viral infection, adding to previous evidence of possible feedbacks between diet and disease in honey bees \cite{AdamTothReview}. 

Moreover, this study also demonstrated the benefits of using data visualizations and multiple datasets to address inherently messy biological data. For instance, by verifying the substantial overlap in our DEG lists to those obtained in another study that addressed a similar question using specimens with less genetic variability, we were able to place much higher confidence in the differential gene expression results from our otherwise noisy data. We also suggested that comparing results derived from polyandrous colony designs (which are usually more natural) and single-drone colony designs (which usually have more signal) may allow researchers to identify transcriptomic patterns in honey bees that are concurrently more realistic and less noisy. Altogether, we hope our results underline the merits of using data visualization techniques and multiple datasets to understand and interpret RNA-sequencing datasets.

\section*{Methods}

\subsection*{Pathogen response}

Details of the procedures we used to prepare virus inoculum, infect and feed caged honey bees, and quantify IAPV can be reviewed in our previous work \cite{adamInt, carrillo}. A linear mixed effects model was used to relate the mortality rates and IAPV titers to the main and interaction effects of the diet and virus factors. The model was fitted to the data by restricted maximum likelihood (REML) using the ``lme'' function in the R package ``nlme''. A random (intercept) effect for experimental setup was included in the model. Post-hoc pairwise comparisons of the four (diet and virus combination) treatment groups were performed and Benjamini-Hochberg adjusted p-values were calculated to limit familywise Type I error rates \cite{bh}.

\subsection*{Design of two-factor experiment}

There are several reasons why, in the current study, we focused only on diet quality (monofloral diets) as opposed to diet diversity (monofloral diets versus polyfloral diets). First, when assessing diet diversity, a sugar diet is often used as a control. However, such an experimental design does not reflect real-world conditions for honey bees as they rarely face a total lack of pollen \cite{DiPasquale}. Second, in studies that compared honey bee health using monofloral and polyfloral diets at the same time, if the polyfloral diet and one of the high-quality monofloral diets both exhibited similarly beneficial effects, then it was difficult for the authors to assess if the polyfloral diet was better than most of the monofloral diets because of its diversity or because it contained as a subset the high-quality monofloral diet \cite{DiPasquale}. Third, as was previously mentioned, honey bees are now confronted with less diverse sources of pollen. As a result, there is a need to better understand how monofloral diets affect honey bee health.

Consequently, for our nutrition factor, we examined two monofloral pollen diets, Rockrose (Cistus) and Castanea (Chestnut). Rockrose pollen is generally considered less nutritious than Chestnut pollen due to its lower levels of protein, amino acids, antioxidants, calcium, and iron \cite{DiPasquale, adamInt}. For our virus factor, one level contained bees that were infected with IAPV and another level contained bees that were not infected with IAPV. This experimental design resulted in four treatment groups (Rockrose pollen without IAPV exposure, Chestnut pollen without IAPV exposure, Rockrose pollen with IAPV exposure, and Chestnut pollen with IAPV exposure) that allowed us to assess main effects and interactive effects between diet quality and IAPV infection in honey bees.

\subsection*{RNA extraction}

Fifteen cages per treatment were originally produced for monitoring of mortality. From these, six live honey bees were randomly selected from each cage 36 hours post inoculation and placed into tubes \cite{carrillo}. Tubes were kept on dry ice and then transferred into a -80C freezer until processing. From the fifteen possible cages, eight were randomly selected for RNA-sequencing. From these eight cages, two of the honey bees per cage were randomly selected from the original six live honey bees per cage. These two bees were combined to form a pooled sample representing the cage. Whole body RNA from each pool was extracted using Qiagen RNeasy MiniKit followed by Qiagen DNase treatment. Samples were suspended in water to 200-400 ng/$\mu$l. All samples were then tested on a Bioanalyzer at the Iowa State University DNA Facility to ensure quality (RIN $>$ 8).

\subsection*{Gene expression}

Samples were sequenced starting on January 14, 2016 at the Iowa State University DNA Facility (Platform: Illumina HiSeq Sequencing; Category: Single End 100 cycle sequencing). A standard Illumina mRNA library was prepared by the DNA facility. Reads were aligned to the BeeBase Version 3.2 genome \cite{hbGenome} from the Hymenoptera Genome Database \cite{hymenopteraDB} using the programs GMAP and GSNAP \cite{gsnap}. There were four lanes of sequencing with 24 samples per lane. Each sample was run twice. Approximately 75-90\% of reads were mapped to the honey bee genome. Each lane produced around 13 million single-end 100 basepair reads.

We tested all six pairwise combinations of treatments for DEGs (pairwise DEGs). We also tested the diet main effect (diet DEGs), virus main effect (virus DEGs), and interaction term for DEGs (interaction DEGs). We then also tested for virus main effect DEGs (virus DEGs) in public data derived from a previous study exploring the gene expression of IAPV virus infection in honey bees \cite{galbraith}. We tested each DEG analysis using recommended parameters with DESeq2 \cite{deseq2}, edgeR \cite{edger}, and LimmaVoom \cite{limma}. In all cases, we used a false discovery rate (FDR) threshold of 0.05 \cite{benjamini}. Fisher's exact test was used to determine significant overlaps between DEG sets (whether from the same dataset but across different analysis pipelines or from different datasets across the same analysis pipelines). The eulerr shiny application was used to construct Venn diagram overlap images \cite{euler}. In the end, we focused on the DEG results from DESeq2 \cite{deseq2} as this pipeline was also used in the Galbraith study \cite{galbraith}. We used the independent filtering process built into the DESeq2 software that mitigates multiple comparison corrections on genes with no power rather than defining one filtering threshold.

\subsection*{Comparison to prior studies on transcriptomic response to viral infection}

We compare the main effect of IAPV exposure in our dataset to that obtained in a previous study conducted by Galbraith and colleagues \cite{galbraith} who also addressed honey bee transcriptomic responses to virus infection. We applied the same downstream bioinformatics analyses between our count table and the count table provided in the Galbraith study. When we applied our bioinformatics pipeline to the Galbraith count table, we obtained different differential expression counts compared to the results published in the Galbraith study. However, there was substantial overlap and we considered this justification to use the differential expression list we obtained in order to keep the downstream bioinformatics analyses as similar as possible between the two datasets (Additional file 17). 

While our study examines honey bees from polyandrous colonies, the Galbraith study examined honey bees from single-drone colonies. As a consequence, the honey bees in our study will be on average 25\% genetically identical, whereas honey bees from the Galbraith study will be on average 75\% genetically identical \cite{sisters}. We note that the difference between these studies may be even greater than this as we used naturally mated honey bees from 15 different colonies. We should therefore expect that the Galbraith study may generate data with lower signal:to:noise ratios than our data due to the lower genetic variation between its replicates. At the same time, our honey bees will be more likely to display the health benefits gained from increased genotypic variance within colonies, including decreased parasitic load \cite{multParasite}, increased tolerance to environmental changes \cite{divHyp2}, and increased colony performance \cite{geneticDiverse, geneticDiverse2}. Given that honey bees are naturally very polyandrous \cite{patriline}, our honey bees may also reflect more realistic environmental and genetic simulations. Taken together, each study provides a different point of value: Our study likely presents less artificial data while the Galbraith data likely presents less messy data. We wish to explore how the gene expression effects of IAPV inoculation compare between these two studies that used such different experimental designs. To achieve this objective, we use visualization techniques to assess the signal:to:noise ratio between these two datasets, and differential gene expression (DEG) analyses to determine any significantly overlapping genes of interest between these two datasets. It is our hope that this aspect of our study may shine light on how experimental designs that control genetic variability to different extents might affect the resulting gene expression data in honey bees.

\subsection*{Visualization}

We used an array of visualization tools as part of our analysis. We first used well-known tools like the PCA plot \cite{pca} from the DESeq2 package. After that, we used lesser-known multivariate visualization tools from our work-in-progress R package called bigPint. Specifically, we used parallel coordinate plots \cite{origPCP}, scatterplot matrices \cite{scatMat}, and litre plots (which we recently developed based on ``replicate line plots'' \cite{jds} (cite bigPint too)) to assess the variability between the replicates and the treatments in our data. We also used these plotting techniques to assess for normalization problems and other common problems in RNA-sequencing analysis pipelines \cite{jds} (cite bigPint too).

We also used statistical graphics to better understand patterns in our DEGs. However, in cases of large DEG lists, these visualization tools had overplotting problems (where multiple objects are drawn on top of one another, making it impossible to detect individual values). To remedy this problem, we first standardized each DEG to have a mean of zero and standard deviation of unity \cite{Chandrasekhar, deSouto}. Then, we performed hierarchical clustering on the standardized DEGs using Ward's linkage. This process divided large DEG lists into smaller clusters of similar patterns, which allowed us to more efficiently visualize the different types of patterns within large DEG lists (see Figures 3 and 4 for examples).

\subsection*{Gene ontology}

DEGs were uploaded as a background list to DAVID Bioinformatics Resources 6.7 \cite{davidBio, davidBio2}. The overrepresented gene ontology (GO) terms of DEGs were determined using the BEEBASE\_ID identifier option (honey bee gene model) in the DAVID software. To fine-tune the GO term list, only terms correlating to Biological Processes were considered. The refined GO term list was then imported into REVIGO \cite{revigo}, which uses semantic similarity measures to cluster long lists of GO terms.

\subsection*{Probing tolerance versus resistance}

To investigate whether the protective effect of good diet is due to direct, specific effects on immune function (resistance), or if it is due to indirect effects of good nutrition on energy availability and vigor (tolerance), we created contrasts of interest (Table \ref{tbl:contrasts}). In particular, we assigned ``resistance candidate DEGs'' to be the ones that were upregulated in the Chestnut group within the virus infected bees but not upregulated in the Chestnut group within the non-infected bees. Our interpretation of these genes is that they represent those that are only activated in infected bees that are fed a high quality diet. We also assigned ``tolerance candidate DEGs'' to be the ones that were upregulated in the Chestnut group for both the virus infected bees and non-infected bees. Our interpretation of these genes is that they represent those that are constitutively activated in bees fed a high quality diet, regardless of whether they are experiencing infection or not. We then determined how many genes fell into these two categories and analyzed their GO terminologies.

\subsection*{Post hoc analysis}

We found considerable noisiness in our data and saw, through gene-level visualizations, that our DEGs contained outliers and inconsistent replicates. Hence, we wanted to explore whether our DEG read counts correlated with pathogen response metrics, including IAPV titers, Schmallenberg Virus (SBV) titers, and mortality rates. For this process, we considered virus main effect DEGs (Figure 4), ``tolerance candidate'' DEGs (Additional file 15), and ``resistance candidate'' DEGs (Additional file 16). For each DEG in each cluster, we calculated a coefficient of determination (R-squared) value to estimate the correlation between its raw read counts and the pathogen response metrics across its 24 samples. We then used the Kruskal–Wallis test to determine if the distribution of the R-squared values in any of the DEG clusters significantly differed from those in the non-DEG genes (the rest of the data). As there were four clusters for each of the nine combinations of DEG lists (``tolerance'' candidate DEGs, ``resistance'' candidate DEGs, and virus-related DEGs) and pathogen response measurements (IAPV titer, SBV titer, and mortality rate), this process resulted in 36 statistical tests.

\end{doublespacing} % Lindsay added

%%%%%%%%%%%%%%%%%%%%%%%%%%%%%%%%%%%%%%%%%%%%%%
%%                                          %%
%% Backmatter begins here                   %%
%%                                          %%
%%%%%%%%%%%%%%%%%%%%%%%%%%%%%%%%%%%%%%%%%%%%%%

\begin{backmatter}

\section*{Ethics approval and consent to participate}
  All honey bees used in this work were sampled in the United States, and no ethical use approval is required for this species in this country.

\section*{Consent for publication}
  Not applicable.

\section*{Availability of data and materials}
  The datasets generated and/or analysed during the current study are available in the [NAME] repository, [PERSISTENT WEB LINK TO DATASETS]. Include our data, Galbraith data, scripts to reproduce tables and figures (on GitHub).

\section*{Competing interests}
  The authors declare that they have no competing interests.

\section*{Funding}
  This work was supported by the United States Department of Agriculture, Agriculture and Food Research Initiative (USDA-AFRI) 2011-04894.

\section*{Author's contributions}
  LR performed the bioinformatic and statistical analyses, produced the figures and tables, and drafted the manuscript. BB conceptualized the study and critically revised the manuscript. AD contributed to experimental design, carried out the laboratory experiments, and processed samples for virus titers and RNA-seq. 
    %All authors read and approved the final manuscript.
    
\section*{Acknowledgements}
  We would like to thank Giselle Narvaez for assisting with cage experiments.
  
%%%%%%%%%%%%%%%%%%%%%%%%%%%%%%%%%%%%%%%%%%%%%%%%%%%%%%%%%%%%%
%%                  The Bibliography                       %%
%%                                                         %%
%%  Bmc_mathpys.bst  will be used to                       %%
%%  create a .BBL file for submission.                     %%
%%  After submission of the .TEX file,                     %%
%%  you will be prompted to submit your .BBL file.         %%
%%                                                         %%
%%                                                         %%
%%  Note that the displayed Bibliography will not          %%
%%  necessarily be rendered by Latex exactly as specified  %%
%%  in the online Instructions for Authors.                %%
%%                                                         %%
%%%%%%%%%%%%%%%%%%%%%%%%%%%%%%%%%%%%%%%%%%%%%%%%%%%%%%%%%%%%%

% if your bibliography is in bibtex format, use those commands:
\bibliographystyle{bmc-mathphys} % Style BST file (bmc-mathphys, vancouver, spbasic).
\bibliography{bmc_honeybee}      % Bibliography file (usually '*.bib' )
% for author-year bibliography (bmc-mathphys or spbasic)
% a) write to bib file (bmc-mathphys only)
% @settings{label, options="nameyear"}
% b) uncomment next line
%\nocite{label}

% or include bibliography directly:
% \begin{thebibliography}
% \bibitem{b1}
% \end{thebibliography}

%%%%%%%%%%%%%%%%%%%%%%%%%%%%%%%%%%%
%%                               %%
%% Figures                       %%
%%                               %%
%% NB: this is for captions and  %%
%% Titles. All graphics must be  %%
%% submitted separately and NOT  %%
%% included in the Tex document  %%
%%                               %%
%%%%%%%%%%%%%%%%%%%%%%%%%%%%%%%%%%%

%%
%% Do not use \listoffigures as most will included as separate files
\newpage
\section*{Figures}

\begin{figure}[h!]
\caption{\csentence{Mortality rates for the four treatment groups, two virus groups, and two diet groups.}
Left to right: Mortality rates for the four treatment groups, two virus groups, and two diet groups. ``N'' represents non-inoculation, ``V'' represents viral inoculation, ``C'' represents Chestnut pollen, and ``R'' represents Rockrose pollen. The mortality rate data included 59 samples with 15 replicates per treatment group, except for the ``NC'' group having 14 replicates. ANOVA values and p-values for the statistical tests are listed in the text of the paper. The letters above the bars represent significant differences with a confidence level of 95\%.}
\end{figure}

\begin{figure}[h!]
\caption{\csentence{IAPV titers for the four treatment groups, two virus groups, and two diet groups.}
Left to right: IAPV titers for the four treatment groups, two virus groups, and two diet groups. ``N'' represents non-inoculation, ``V'' represents viral inoculation, ``C'' represents Chestnut pollen, and ``R'' represents Rockrose pollen. The IAPV titer data included 38 samples with 10 replicates per treatment group, except for the ``NR'' group having 8 replicates. ANOVA values and p-values for the statistical tests are listed in the text of the paper. The letters above the bars represent significant differences with a confidence level of 95\%.}
\end{figure}

\begin{figure}[h!]
\caption{\csentence{Parallel coordinate plots of the 1,019 DEGs after hierarchical clustering of size four between the virus-infected and control groups of the Galbraith data \cite{galbraith}.}
Parallel coordinate plots of the 1,019 DEGs after hierarchical clustering of size four between the virus-infected and control groups of the Galbraith study. ``N'' represents non-inoculation, ``V'' represents viral inoculation. Clusters 1, 2, and 4 seem to represent DEGs that were overexpressed in the virus inoculated group, and Cluster 3 seems to represent DEGs that were overexpressed in the non-inoculated control group. In general, the DEGs appeared as expected, but there is rather noticeable deviation of the first replicate from the virus-treated sample (``V.1'') from the other virus-treated replicates in Cluster 1.}
\end{figure}

\begin{figure}[h!]
\caption{\csentence{Parallel coordinate plots of the 43 DEGs after hierarchical clustering of size four between the virus-infected and control groups of our study.}
Parallel coordinate plots of the 43 DEGs after hierarchical clustering of size four between the virus-infected and control groups of our study. ``N'' represents non-infected control group, and ``V'' represents treatment of virus. The vertical red line indicates the distinction between treatment groups. We see from this plot that the DEG designations for this dataset do not appear as clean compared to what we saw in the Galbraith dataset in Figure 3.}
\end{figure}

\begin{figure}[h!]
\caption{\csentence{Gene ontology analysis results for the 122 DEGs related to our ``tolerance'' hypothesis and for the 125 DEGs related to our ``resistance'' hypothesis.}
GO analysis results for the 122 DEGs related to our ``tolerance'' hypothesis (A) and for the 125 DEGs related to our ``resistance'' hypothesis (B). The color and size of the circles both represent the number of genes in that ontology. The x-axis and y-axis are organized by SimRel, a semantic similarity metric \cite{semantic}.}
\end{figure}

\begin{figure}[h!]
\caption{\csentence{Venn diagrams comparing the virus-related DEG overlaps between our dataset and the Galbraith dataset.}
Venn diagrams comparing the virus-related DEG overlaps between the Galbraith study (labeled as ``G'') and our study (labeled as ``R''). From left to right: Total virus-related DEGs (subplot A), virus-upregulated DEGs (subplot B), control-upregulated DEGs (subplot C). Both the total virus-related and virus-upregulated DEGs showed significant overlap between the studies (p-value $<$ 2.2e-16) as per Fisher's Exact Test for Count Data. There was one gene that was virus-upregulated in the Galbraith study but control-upregulated in our study.}
\end{figure}

\newpage
\section*{Tables}

\begin{table}[h!]
\begin{tabular}{ccccc}
\hline
BeeBase ID & Gene Name & Known functions & Us & Galbraith \\ 
\hline
GB41545 & \makecell{MD-2-related \\ lipid-recognition \\ protein-like} & \makecell{Implicated in lipid recognition, \\ particularly in the recognition of \\ pathogen related products} & N & - \\
\hline
GB50955 & \makecell{Protein \\ argonaute-2} & \makecell{Interacts with small interfering RNAs \\ to form RNA-induced silencing \\ complexes which target and cleave \\ transcripts that are mostly from \\ viruses and transposons} & V & V \\
\hline
GB48755 & \makecell{UBA-like \\ domain-containing \\ protein 2} & \makecell{Found in diverse proteins involved \\ in ubiquitin/proteasome \\ pathways} & V & V \\
\hline
GB47407 & Histone H4 & \makecell{Capable of affecting transcription, \\ DNA repair, and DNA replication \\ when post-transcriptionally modified} & V & V \\
\hline
GB42313 & \makecell{Leishmanolysin-like \\ peptidase} & \makecell{Encodes a protein involved in cell \\ migration and invasion; implicated in \\ mitotic progression in D. melanogaster} & V & V \\
\hline
GB50813 & \makecell{Rho guanine \\ nucleotide \\ exchange factor 11} & \makecell{Implicated in regulation of apoptopic \\ processes, cell growth, signal \\ transduction, and transcription} & V & V \\
\hline
GB54503 & \makecell{Thioredoxin \\ domain-containing \\ protein} & \makecell{Serves as a general protein \\ disulphide oxidoreductase} & N & - \\
\hline
GB53500 & \makecell{Transcriptional \\ regulator Myc-B} & \makecell{Regulator gene that codes for a \\ transcription factor} & V & V \\
\hline
GB51305 & Tropomyosin-like & \makecell{Related to protein involved in muscle \\ contraction} & N & N \\
\hline
GB50178 & \makecell{Cilia and \\ flagella-associated \\ protein 61-like} & \makecell{Induces components required for \\ wild-type motility and \\ stable assembly of motile cilia} & V & V \\
\hline
\end{tabular}
\caption{Known functions of the mapped subset of 43 DEGs in the virus main effect of our study. Whether the gene was overrepresented in the virus or non-virus group is also indicated for both our study and the Galbraith study. Functionalities were extracted from Flybase, National Center for Biotechnology Information and The European Bioinformatics Institute databases.}
  \label{tbl:virusGenes}
\end{table}

\begin{table}[h!]
\begin{tabular}{cccc}
\hline
Contrast & DEGs & Interpretation & Results \\ 
\hline
V (all) vs N (all) & 43 & \makecell{Genes that change expression \\ due to virus effect regardless \\ of diet status in bees} & Table \ref{tbl:virusGenes} \\
\hline
NC vs NR & 941 & \makecell{Genes that change expression \\ due to diet effect in \\ uninfected bees} & \makecell{Supplementary \\ tables 4 and 5, \\ Additional file 1} \\
\hline
VC vs VR & 376 & \makecell{Genes that change expression \\ due to diet effect in \\ infected bees} & \makecell{Supplementary \\ tables 6 and 7, \\ Additional file 1} \\
\hline
\makecell{VC upregulated in VC vs VR, and \\ NC upregulated in NC vs NR} & 122 & \makecell{``Tolerance'' genes that turn \\ on by good diet regardless of \\ virus infection status in bees} & Figure 5A \\
\hline
\makecell{VC upregulated in VC vs VR, but \\ NC not upregulated in NC vs NR} & 125 & \makecell{``Resistance'' genes that turn \\ on by good diet only in \\ infected bees} & Figure 5B \\
\hline
\end{tabular}
\caption{Contrasts in our study for assessing GO and pathways analysis.}
  \label{tbl:contrasts}
\end{table}

\newpage
\section*{Additional Files}

  \subsection*{Additional file 1 --- Supplementary tables.}
    \textbf{Table 1:} Number of DEGs across three analysis pipelines for (A) the diet main effect in our study, (B) the virus main effect in our study, and (C) the virus main effect in the Galbraith study. For the diet effects, ``C'' represents Chestnut diet and ``R'' represents Rockrose diet. For the virus effects, ``N'' represents control non-inoculated and ``V'' represents virus-inoculated.
    \textbf{Table 2:} Pathways related to the 1,033 DEGs that were upregulated in the Chestnut treatment from the diet main effect.
    \textbf{Table 3:} Pathways related to the 881 DEGs that were upregulated in the Rockrose treatment from the diet main effect.
    \textbf{Table 4:} GO analysis results for the 601 DEGs that were upregulated in the NC treatment from the NC versus NR treatment pair analysis. These DEGs represent genes that are upregulated when non-infected honey bees are given high quality Chestnut pollen compared to being given low quality Rockrose pollen.
    \textbf{Table 5:} GO analysis results for the 340 DEGs that were upregulated in the NR treatment from the NC versus NR treatment pair analysis. These DEGs represent genes that are upregulated when non-infected honey bees are given low quality Rockrose pollen compared to being given high quality Chestnut pollen.
    \textbf{Table 6:} GO analysis results for the 247 DEGs that were upregulated in the VC treatment from the VC versus VR treatment pair analysis. These DEGs represent genes that are upregulated when infected honey bees are given high quality Chestnut pollen compared to being given low quality Rockrose pollen.
    \textbf{Table 7:} GO analysis results for the 129 DEGs that were upregulated in the VR treatment from the VC versus VR treatment pair analysis. These DEGs represent genes that are upregulated when infected honey bees are given low quality Rockrose pollen compared to being given high quality Chestnut pollen.
    \textbf{Table 8:} Number of DEGs across three analysis pipelines for all six treatment pair combinations between the diet and virus factor. ``C'' represents Chestnut diet, ``R'' represents Rockrose diet, ``V'' represents virus-inoculated, and ``N'' represents control non-inoculated.
    \textbf{Table 9:} Kruskal-Wallis p-value and Bonferroni corrections for the 36 combinations of DEG lists, pathogen response metrics, and cluster number. (XLS).

  \subsection*{Additional file 2 --- PCA plots for the Galbraith dataset and for our dataset.}
    PCA plots for the Galbraith dataset (A) and for our dataset (B). ``V'' represents virus-inoculated, and ``N'' represents control non-inoculated. The x-axis represents the principal component with the most variation and the y-axis represents the principal component with the second-most variation (PNG).

  \subsection*{Additional file 3 --- Parallel coordinate lines of the diet-related DEGs of our dataset.}
    Parallel coordinate plots of the 1,914 DEGs after hierarchical clustering of size six between the Chestnut and Rockrose groups of our study. Here ``C'' represents Chestnut samples, and ``R'' represents Rockrose samples. The vertical red line indicates the distinction between treatment groups. We see from this plot that the DEG designations for this dataset do not appear as clean compared to what we saw in the Galbraith dataset in Figure 3 (PNG).

  \subsection*{Additional file 4 --- Example litre plots from the virus-related DEGs of our dataset.}
    Example litre plots of the nine DEGs with the lowest FDR values from the 43 virus-related DEGs of our dataset. ``N'' represents non-infected control samples and ``V'' represents virus-treated samples. Most of the magenta points (representing the 144 combinations of samples between treatment groups for a given DEG) do not reflect the expected pattern as clearly compared to what we saw in the litre plots of the Galbraith data. They are not as clustered together (representing replicate inconsistency) and they sometimes cross the \textit{x=y} line (representing lack of difference between treatment groups). This finding reflects what we saw in the messy looking parallel coordinate lines of Figure 4 (PNG).

  \subsection*{Additional file 5 --- Example litre plots of DEGs from Cluster 1 of the Galbraith dataset.}
    Example litre plots of the nine DEGs with the lowest FDR values from the 365 DEGs in Cluster 1 (originally shown in Figure 3) of the Galbraith dataset. ``N'' represents non-infected control samples and ``V'' represents virus-treated samples. Most of the light orange points (representing the nine combinations of samples between treatment groups for a given DEG) deviate from the \textit{x=y} line in a tight bundle as expected (PNG).

  \subsection*{Additional file 6 --- Example litre plots of DEGs from Cluster 2 of the Galbraith dataset.}
    Example litre plots of the nine DEGs with the lowest FDR values from the 327 DEGs in Cluster 2 (originally shown in Figure 3) of the Galbraith dataset. ``N'' represents non-infected control samples and ``V'' represents virus-treated samples. Most of the dark orange points (representing the nine combinations of samples between treatment groups for a given DEG) deviate from the \textit{x=y} line in a compact clump as expected. However, they are not as tightly bunched together compared to what we saw in the example litre plots of Cluster 1 (shown in Additional file 5). As a result, what we see in these litre plots reflects what we saw in the parallel coordinate lines of Figure 3: The replicate consistency in the Cluster 1 DEGs is not as clean as that in the Cluster 2 DEGs, but is still relatively clean (PNG).

  \subsection*{Additional file 7 --- Scatterplot matrix of DEGs from Cluster 1 of the Galbraith dataset.}
    The 365 DEGs from the first cluster of the Galbraith dataset (originally shown in Figure 3) superimposed as light orange dots onto all genes as black dots in the form of a scatterplot matrix. The data has been standardized. ``N'' represents non-infected control samples and ``V'' represents virus-treated samples. We confirm that the DEGs mostly follow the expected structure, with their placement deviating from the \textit{x=y} line in the treatment scatterplots, but adhering to the \textit{x=y} line in the replicate scatterplots. However, we do see that sample ``V.1'' may be somewhat inconsistent in these DEGs, as its presence in the replicate scatterplots shows DEGs deviating from the \textit{x=y} line more than expected and its presence in the treatment scatterplots shows DEGs adhering to the \textit{x=y} line more than expected. This inconsistent sample was something we observed in Figure 3 (PNG).
    
  \subsection*{Additional file 8 --- Scatterplot matrix of DEGs from Cluster 2 of the Galbraith dataset.}
    The 327 DEGs from the second cluster of the Galbraith dataset (originally shown in Figure 3) superimposed as dark orange dots onto all genes as black dots in the form of a scatterplot matrix. The data has been standardized. ``N'' represents non-infected control samples and ``V'' represents virus-treated samples. We confirm that the DEGs mostly follow the expected structure, with their placement deviating from the \textit{x=y} line in the treatment scatterplots, but adhering to the \textit{x=y} line in the replicate scatterplots (PNG).
    
  \subsection*{Additional file 9 --- Scatterplot matrix of DEGs from Cluster 3 of the Galbraith dataset.}
    The 224 DEGs from the third cluster of the Galbraith dataset (originally shown in Figure 3) superimposed as turquoise dots onto all genes as black dots in the form of a scatterplot matrix. The data has been standardized. ``N'' represents non-infected control samples and ``V'' represents virus-treated samples. We confirm that the DEGs mostly follow the expected structure, with their placement deviating from the \textit{x=y} line in the treatment scatterplots, but adhering to the \textit{x=y} line in the replicate scatterplots (PNG).
    
  \subsection*{Additional file 10 --- Scatterplot matrix of DEGs from Cluster 4 of the Galbraith dataset.} 
    The 103 DEGs from the fourth cluster of the Galbraith dataset (originally shown in Figure 3) superimposed as pink dots onto all genes as black dots in the form of a scatterplot matrix. The data has been standardized. ``N'' represents non-infected control samples and ``V'' represents virus-treated samples. We confirm that the DEGs mostly follow the expected structure, with their placement deviating from the \textit{x=y} line in the treatment scatterplots, but adhering to the \textit{x=y} line in the replicate scatterplots. We also see that the second replicate from the virus-treated sample (``V.2'') may be somewhat inconsistent in these DEGs, as its presence in the replicate scatterplots results in the DEGs unexpectedly deviating from the \textit{x=y} line and its presence in the treatment scatterplots results in the DEGs unexpectedly adhering to the \textit{x=y} line (PNG).
    
  \subsection*{Additional file 11 --- Scatterplot matrix of virus-related DEGs from our dataset, showing only replicates 1, 2, and 3.} 
    The 43 virus-related DEGs from our dataset superimposed as magenta dots onto all genes in the form of a scatterplot matrix. Only replicates 1, 2, and 3 are shown from both treatment groups. The data has been standardized. ``N'' represents non-infected control samples and ``V'' represents virus-treated samples. We see that, compared to the scatterplot matrices from certain clusters of the Galbraith data, the 43 DEGs from this subset of six samples from our data do not paint as clear of a picture, sometimes unexpectedly deviating from the \textit{x=y} line in the replicate plots and sometimes unexpectedly adhering to the \textit{x=y} line in the treatment plots (PNG).
    
  \subsection*{Additional file 12 --- Scatterplot matrix of virus-related DEGs from our dataset, showing only replicates 4, 5, and 6.}     
    The 43 virus-related DEGs from our dataset superimposed as magenta dots onto all genes in the form of a scatterplot matrix. Only replicates 4, 5, and 6 are shown from both treatment groups. The data has been standardized. ``N'' represents non-infected control samples and ``V'' represents virus-treated samples. We see that, compared to the scatterplot matrices from certain clusters of the Galbraith data, the 43 DEGs from this subset of six samples from our data do not paint as clear of a picture, and most of them unexpectedly adhere to the \textit{x=y} line in the treatment plots (PNG).
    
  \subsection*{Additional file 13 --- Scatterplot matrix of virus-related DEGs from our dataset, showing only replicates 7, 8, and 9.}    
    The 43 virus-related DEGs from our dataset superimposed as magenta dots onto all genes in the form of a scatterplot matrix. Only replicates 7, 8, and 9 are shown from both treatment groups. The data has been standardized. ``N'' represents non-infected control samples and ``V'' represents virus-treated samples. We see that, compared to the scatterplot matrices from certain clusters of the Galbraith data, the 43 DEGs from this subset of six samples from our data do not paint as clear of a picture, sometimes unexpectedly deviating from the \textit{x=y} line in the replicate plots and sometimes unexpectedly adhering to the \textit{x=y} line in the treatment plots (PNG).
    
  \subsection*{Additional file 14 --- Scatterplot matrix of virus-related DEGs from our dataset, showing only replicates 10, 11, and 12.}    
    The 43 virus-related DEGs from our dataset superimposed onto all genes in the form of a scatterplot matrix. Only replicates 10, 11, and 12 are shown from both treatment groups. The data has been standardized. ``N'' represents non-infected control samples and ``V'' represents virus-treated samples. We see that, compared to the scatterplot matrices from certain clusters of the Galbraith data, the 43 DEGs from this subset of six samples from our data do not paint as clear of a picture, and most of them unexpectedly deviate from the \textit{x=y} line in the virus-related replicate plots (PNG).
    
   \subsection*{Additional file 15 --- Parallel coordinate plots of the ``tolerance'' candidate DEGs.}    
    Parallel coordinate plots of the 122 DEGs after hierarchical clustering of size four between the ``tolerance'' candidate DEGs. Here ``N'' represents non-infected control group, ``V'' represents treatment of virus, ``C'' represents high-quality Chestnut diet, and ``R'' represents low-quality Rockrose diet. The vertical red line indicates the distinction between treatment groups. We see there is considerable noise in the data (non-consistent replicate values), but that the general patterns of the DEGs follow what we expect based on our ``tolerance'' contrast (PNG).
    
   \subsection*{Additional file 16 --- Parallel coordinate plots of the ``resistance'' candidate DEGs.}    
    Parallel coordinate plots of the 125 DEGs after hierarchical clustering of size four between the ``resistance'' candidate DEGs. Here ``N'' represents non-infected control group, ``V'' represents treatment of virus, ``C'' represents high-quality Chestnut diet, and ``R'' represents low-quality Rockrose diet. The vertical red line indicates the distinction between treatment groups. We see there is considerable noise in the data (non-consistent replicate values), but that the general patterns of the DEGs follow what we expect based on our ``resistance'' contrasts (PNG).

   \subsection*{Additional file 17 --- Venn diagrams comparing the virus-related DEG overlaps in the Galbraith data using our pipeline and the pipeline used by Galbraith et al.}
  Venn diagrams comparing the virus-related DEG overlaps of the Galbraith data from the DESeq2 bioinformatics pipelines used in the Galbraith study (labeled as ``G.O.'') and the DESeq2 bioinformatics pipelines used in our study (labeled as ``G.R''). While we were not able to fully replicate the DEG list published in the Galbraith study, our DEG list maintained significant overlaps with their DEG list. From left to right: Total virus-related DEGs (subplot A), virus-upregulated DEGs (subplot B), control-upregulated DEGs (subplot C) (PNG).
    
   \subsection*{Additional file 18 --- Venn diagrams of main effect DEG overlaps across DESeq2, edgeR, and limma}
  Venn diagrams comparing DEG overlaps across DESeq2, edgeR, and limma for our diet main effect (top row), our virus main effect (middle row), and the Galbraith virus main effect (bottom row). Within a given subplot, ``D'' represents DESeq2, ``E'' represents edgeR, and ``L'' represents limma. From left to right on top row: Total diet-related DEGs (subplot A), Castanea-upregulated DEGs (subplot B), Rockrose-upregulated DEGs (subplot C). From left to right on middle row: Total virus-related DEGs (subplot D), virus-upregulated DEGs (subplot E), control-upregulated DEGs in our data (subplot F). From left to right on bottom row: Total virus-related DEGs (subplot G), virus-upregulated DEGs (subplot H), control-upregulated DEGs in the Galbraith data (subplot I) (PNG). With the exception of the limma pipeline resulting in zero DEGs in our virus main effect analysis, we found significant overlaps between DEG lists across the different pipelines (DESeq2, edgeR, and limma). In general, DESeq2 resulted in the largest number of DEGs and limma resulted in the least number of DEGs (PNG). 
  
  \subsection*{Additional file 19 --- Analysis of correlation between DEG read counts and pathogen response metrics}
  Distribution of R-squared values for DEG cluster read counts and pathogen response metrics. Columns left to right: SBV titers, mortality rates, and IAPV titers. Rows top to bottom: Tolerance candidate DEGs, resistance candidate DEGs, and virus-related DEGs. Each subplot includes five boxplots which represent the R-squared value distributions for four DEG clusters and all remaining non-DEGs in the data. The top number above each boxplot represents the number of genes included. The first four boxplots also include a bottom number, which represents the Kruskal–Wallis p-value of the comparison of the R-squared distribution of the cluster and the R-squared distribution of the non-DEG data (PNG).
  
\end{backmatter}
\end{linenumbers} % Lindsay added
\end{document}
